\documentclass[11pt]{article}
\usepackage{mathptmx}
% \usepackage{mathpazo} %Like Palatino with extensive math support
\usepackage{fullpage}
\usepackage{amsmath}
\usepackage[normalem]{ulem}
\usepackage[sort,comma,round,authoryear]{natbib}
\linespread{1.7}
\usepackage{graphicx}
\usepackage[utf8]{inputenc}
\usepackage{lineno}
% \usepackage{titlesec}
% \titleformat{\section}[block]{\Large\bfseries\filcenter}{\thesection}{1em}{}
% \titleformat{\subsection}[block]{\Large\itshape\filcenter}{\thesubsection}{1em}{}
% \titleformat{\subsubsection}[block]{\large\itshape}{\thesubsubsection}{1em}{}
% \titleformat{\paragraph}[runin]{\itshape}{\theparagraph}{1em}{}[. ]\renewcommand{\refname}{Literature Cited}

\title{Supplementary Material for: The joint evolution of movement and competition strategies}

% This version of the LaTeX template was last updated on
% November 8, 2019.

%%%%%%%%%%%%%%%%%%%%%
% Authorship
%%%%%%%%%%%%%%%%%%%%%
% Please remove authorship information while your paper is under review,
% unless you wish to waive your anonymity under double-blind review. You
% will need to add this information back in to your final files after
% acceptance.

\author{Pratik R. Gupte$^{1,\ast}$ \\ 
        Christoph F. G. Netz$^{1}$ \\ 
        Franz J. Weissing$^{1, \ast}$}

\date{}

\begin{document}

\maketitle

\noindent{} 1. Groningen Institute for Evolutionary Life Sciences, University of Groningen, Groningen 9747 AG, The Netherlands.

\noindent{} $\ast$ Corresponding authors; e-mail: p.r.gupte@rug.nl or f.j.weissing@rug.nl

\bigskip

\newpage

\section{Landscape Depletion at Varying $r_{max}$}

\subsection{Scenario 1: No Kleptoparasitism}

\begin{figure}[h!]
        \centering
        \includegraphics*[width=1.0\textwidth]{figures/fig_landscape_rmax_foragers.png}
        \caption{In scenario 1, foragers completely deplete the resource landscape within 10 generations at low $r_{max}$ (\textbf{A, B}).
        However, at $r_{max} >$ 0.01 (\textbf{C, D}, prey item regeneration exceeds depletion and the resource landscape is rapidly saturated until most cells carry 5 items, the maximum allowed in our model.)}
\end{figure}

\newpage

\subsection{Scenario 2: Fixed Foraging Strategies}

\begin{figure}[h!]
        \centering
        \includegraphics*[width=1.0\textwidth]{figures/fig_landscape_rmax_obligate.png}
        \caption{In scenario 2, foragers can only deplete the resource landscape at very low $r_{max}$ (\textbf{A}; 1 prey item generated per 1,000 timesteps, or 2.5 generations).
        At all $r_{max} \geq$ 0.05 (\textbf{B, C, D}, prey item regeneration matches or exceeds depletion and the resource landscape either shows strong spatial structure, or is entirely saturated with prey items.)}
\end{figure}

\newpage

\subsection{Scenario 3: Conditional Foraging Strategies}

\begin{figure}[h!]
        \centering
        \includegraphics*[width=1.0\textwidth]{figures/fig_landscape_rmax_facultative.png}
        \caption{Scenario 3 is similar to scenario 1 at low $r_{max}$ (\textbf{A, B}, where foragers completely deplete the resource landscape).
        Similarly, at $r_{max} >$ 0.01 (\textbf{C, D}, prey item regeneration exceeds depletion and the resource landscape is rapidly saturated to a carrying capacity of 5 prey items per cell.)}
\end{figure}

\newpage

\section{Evolution of Decision Making Weights}

\subsection{Scenario 1: No Kleptoparasitism}

\begin{figure}[h!]
        \centering
        \includegraphics*[width=0.9\textwidth]{figures/fig_wt_evo_foragers.png}
        \caption{In scenario 1, populations evolve \textbf{(A)} to move away from non-handlers, \textbf{(B)} move towards handlers, and \textbf{(C)} to move towards cells with more prey items.
        While the sign of the response (avoidance or preference) is consistent across replicates, the replicates differ in the number and frequency of evolved responses (i.e., decision-making weight values).
        All panels show an $r_{max}$ = 0.01.}
\end{figure}

\newpage

\subsection{Scenario 2: Fixed Foraging Strategies}

\begin{figure}[h!]
        \centering
        \includegraphics*[width=0.9\textwidth]{figures/fig_wt_evo_obligate.png}
        \caption{In scenario 2, populations evolve \textbf{(A)} to move away from non-handlers, but \textbf{(B)} a mixed response towards handlers due to the correlation between handler-preference and the kleptoparasitic strategy (see Fig. 3, main text).
        Here too, responses are polymorphic, with little consistency across replicates, despite the overall sign of the response being consistent.
        \textbf{(C)} In contrast with scenario 1, most foragers show only a weak preference for moving towards cells with more prey items.
        All panels show an $r_{max}$ = 0.01.}
\end{figure}

\newpage

\subsection{Scenario 3: Conditional Foraging Strategies}

\begin{figure}[h!]
        \centering
        \includegraphics*[width=0.9\textwidth]{figures/fig_wt_evo_facultative.png}
        \caption{In scenario 3, populations evolve \textbf{(A)} to move away from non-handlers, and \textbf{(B)} towards handlers.
        \textbf{(C)} Most foragers also show a weak preference for moving towards cells with more prey items.
        \textbf{(D)} All individuals show a kleptoparasitic response to handlers.
        All panels show an $r_{max}$ = 0.01.}
\end{figure}


\end{document}